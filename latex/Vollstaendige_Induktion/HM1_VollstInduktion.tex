%%%%%%%%%%%%%%%%%%%%%%%%%%%%%%%%%%%%%%%%%
% Contract
% LaTeX Template
% Version 1.0 (December 8 2014)
%
% This template has been downloaded from:
% http://www.LaTeXTemplates.com
%
% Original author:
% Brandon Fryslie
% With extensive modifications by:
% Vel (vel@latextemplates.com)
%
% License:
% CC BY-NC-SA 3.0 (http://creativecommons.org/licenses/by-nc-sa/3.0/)
%
% Note:
% If you are using Apple OS X, go into structure.tex and uncomment the font
% specifications for OS X and comment out the default specifications - this will
% drastically increase how good the document looks. You will now need to
% compile with XeLaTeX.
%
%%%%%%%%%%%%%%%%%%%%%%%%%%%%%%%%%%%%%%%%%

\documentclass[a4paper,12pt]{article} % The default font size is 12pt on A4 paper, change to 'usletter' for US Letter paper and adjust margins in structure.tex
\usepackage{amssymb}
\usepackage{fancyhdr}
\fancyhf{}
\fancyhead[L]{HM1 Tutorium 3}
\fancyhead[R]{08.11.2018}
\fancyfoot[L]{Moritz Luca Schmid}
\fancyfoot[R]{\thepage}
\renewcommand{\headrulewidth}{1pt}
\renewcommand{\footrulewidth}{1pt}
\fancypagestyle{plain}{}
\pagestyle{fancy}
\setlength{\headheight}{25pt}
\addtolength{\headwidth} {23mm}

\input{../structure.tex} % Input the structure.tex file which specifies the document layout and style


\begin{document}
\section{Vollständige Induktion}
Um bei der Vollständigen Induktion nicht den Überblick zu verlieren und alles mathematisch/formal richtig zu machen, ist es wichtig die folgende Schritte streng einzuhalten:

\begin{enumerate}
\item \textbf{Zu Zeigen:} Die Gleichung, die zu Zeigen ist, abschreiben.
\item \textbf{Induktionsanfang:} Zeige die Aussage für n=1
\item \textbf{Induktionsvoraussetzung:} Annahme, dass die zu zeigende Gleichung für ein \textbf{beliebiges, aber festes n} gelte. (Gleichung vom Schritt "Zu Zeigen" abschreiben.)
\item \textbf{Induktionsschritt:} Bei der einen Seite der Gleichung das 'n' mit 'n+1' ersetzen und dann \textbf{mit Hilfe der Induktionsvoraussetzung} zur anderen Seite der Gleichung (mit 'n+1' anstatt 'n') umformen.
\end{enumerate}
Bei der Vollständigen Induktion muss \textbf{in jedem Fall die Induktionsvoraussetzung} genutzt werden!\\
Um die eine Seite der Gleichung im Induktionsschritt in die andere Seite umzuformen ist oftmals ein Blick auf die zweite Seite hilfreich um zu erkennen, wie man die Gleichung umformen muss um auf die gewünschte Form (also die zweite Seite der Gleichung) zu kommen.\\ 
Weitere Tipps sind:
\begin{itemize}
\item Summe $\Sigma$ modulieren (Grenzen verändern, substituieren, anpassen)
\item "Fertige Fragmente" ausklammern (die genau so auch auf der anderen Seite der Gleichung stehen)
\item Polynomdivision
\item auf gleichen Nenner bringen / erweitern
\item Exponenten zusammenfassen / auseinanderziehen
\item Abschätzungen machen (bei Ungleichungen)
\item bei Rekursion mit Vorgänger arbeiten (einsetzen)
\end{itemize}
\end{document}
