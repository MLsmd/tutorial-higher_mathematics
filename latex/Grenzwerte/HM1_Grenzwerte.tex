

\documentclass[a4paper,12pt]{article} % The default font size is 12pt on A4 paper, change to 'usletter' for US Letter paper and adjust margins in structure.tex
\usepackage{amssymb}
\usepackage{fancyhdr}
\fancyhf{}
\fancyhead[L]{HM1 Tutorium 12}
\fancyhead[R]{24.01.2019}
\fancyfoot[L]{Moritz Luca Schmid}
\fancyfoot[R]{\thepage}
\renewcommand{\headrulewidth}{1pt}
\renewcommand{\footrulewidth}{1pt}
\fancypagestyle{plain}{}
\pagestyle{fancy}
\setlength{\headheight}{25pt}
\addtolength{\headwidth} {23mm}

\input{../structure.tex} % Input the structure.tex file which specifies the document layout and style


\begin{document}

\section{Grenzwert}
\subsection{Definition}
\begin{center}
\framebox[100pt]{$lim_{x \rightarrow x_0} \ \ f(x) = y$}
\end{center}
,wenn $\forall \epsilon > 0 \ \ \ \ \exists \delta > 0$, so dass
\begin{center}
\framebox[200pt]{$0 < |x-x_0| < \delta \ \ \Rightarrow |f(x) - y| < \epsilon$}
\end{center}
Diese Definition sagt, dass wenn man sich einem bestimmten $x_0$ annähert (also der Abstand $|x-x_0|$ sehr klein wird), sich auch der y Wert dem Funktionswert an dieser Stelle annähert.\\
Wenn dies nicht der Fall ist, \textbf{existiert der Limes nicht}. Dies ist zum Beispiel bei Sprungstellen der Fall.
\subsection{Trickkiste}
\begin{itemize}
\item \textbf{l'Hopital:} Nenner und Zähler eines Bruches konvergieren \textbf{beide} gegen Null oder $\infty / -\infty$ $\rightarrow$ Grenzwert von der Ableitung des Bruches bilden
\item x aus Nenner/Zähler herausziehen
\item \textbf{Kürzungstrick}
\item \textbf{Sandwichkriterium}
\item Reihendarstellung verwenden $\rightarrow$ Teleskopsummenarstellung, dann kürzen
\item \textbf{Polynomdivision }um auf gleichen Nennen zu bringen / zu kürzen
\item \textbf{Binominalsatz}
\item bei \textbf{3. Wurzel}: $a-b = \frac{a^3-b^3}{a^2+ab+b^2}$ (bei Quadratwurzel 3. Binomische Formel)
\item 2 \textbf{Teilfolgen} mit unterschiedlichem Grenzwert $\rightarrow$ $f(x)$ hat keinen Grenzwert
\item bei Nenner $\rightarrow 0$ (z.B. $lim_{x\rightarrow3} \frac{1}{x-3}$) \textbf{einseitigen Grenzwert} betrachten.
\item Produkt \textbf{in Bruch "zwingen"}; dann z.B. l'Hopital (z.B. $x*ln(x) = \frac{ln(x)}{\frac{1}{x}}$)
\end{itemize}

\end{document}
