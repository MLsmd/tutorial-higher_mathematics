

\documentclass[a4paper,12pt]{article} % The default font size is 12pt on A4 paper, change to 'usletter' for US Letter paper and adjust margins in structure.tex
\usepackage{amssymb} %mathbb{}\usepackage{amssymb} %mathbb{}
\usepackage{fancyhdr} %\framebox[150pt]{}
\usepackage{amsmath,tikz} 
\fancyhf{}
\fancyhead[L]{HM1 Tutorium 7}
\fancyhead[R]{06.12.2018}
\fancyfoot[L]{Moritz Luca Schmid}
\fancyfoot[R]{\thepage}
\renewcommand{\headrulewidth}{1pt}
\renewcommand{\footrulewidth}{1pt}
\fancypagestyle{plain}{}
\pagestyle{fancy}
\setlength{\headheight}{25pt}
\addtolength{\headwidth} {23mm}

\input{../structure.tex} % Input the structure.tex file which specifies the document layout and style


\begin{document}
\section{Lineare Algebra}
\subsection{Linearkombination}
Die Linearkombination ist ein Vektor, der sich durch \textbf{Vektoraddition} und \textbf{skalare Multiplikation} durch andere Vektoren darstellen lässt.
\begin{center}
$lin \{ V_j \} \ \ = \ \ \Sigma_{j=1}^n \alpha_j V_j$
\end{center}
\subsection{Basis}
Die Basis eines Vektorraums U ist die Menge an Vektoren $V_1,V_2,...,V_n$, wenn diese linear unabhängig sind und $lin\{V_1,V_2,...V_n\} \ = \ U$ gilt.\\
Bei einem n-dimensionalen Vektorraum enthält die Basis also n Elemente.\\
Die \textbf{Standartbasis} enthält nur Vektoren, die nur Nullen und eine Eins enthalten. (z.B. $\{(0,1),(1,0)\}$)

\subsection{Zeilenstufenform}

\begin{center}
  $
\left(
   \begin{array}{ccccc}
3 & 1 & 0 & 2 & 3\\
0 & 0 & 5 & 0 & 4\\
0 & 0 & 0 & 1 & 0\\
0 & 0 & 0 & 0 & 0
   \end{array}
\right)
$
  \end{center}  
\subsection{Zeilennormalform}
\begin{center}
  $
\left(
   \begin{array}{ccccc}
1 & 0 & 0 & 0 & 0\\
0 & 1 & 0 & 0 & 0\\
0 & 0 & 0 & 1 & 0\\
0 & 0 & 0 & 0 & 0
   \end{array}
\right)
$
  \end{center}  
\textbf{Algorithmus für Zeilennormalform:}
\begin{enumerate}
\item in ZSF bringen
\item teilen, sodass 1er entstehen
\item met unteren Zeilen die 0er über den 1en erzeugen
\end{enumerate}
\subsection{Kern einer Matrix}
\begin{center}
\framebox[180pt]{$Kern A := \{ \vec{x} \in \mathbb{K}^m \ : \ A \vec{x} = \vec{0} \} \subseteq \mathbb{K}^m$}
\end{center}
\begin{itemize}
\item Kern(A) ist Unterraum von $\mathbb{K}^m$
\item Lösung + Element des Kerns ist auch Lösung ($\rightarrow$ Def. Unterraum)
\end{itemize}
\textbf{Kern bestimmen:}
\begin{enumerate}
\item $A\vec{x}=0$ lösen (Gauß / Zeilennormalform)
\item \textbf{(-1) - Trick} anwenden (siehe 1.2)
\end{enumerate}
\subsection{(-1) - Trick}
Der \textbf{(-1) - Trick} ist eine \textbf{Karlsruher Spezialität}.\\
Vorgehen:
\begin{enumerate}
\item Matrix in \textbf{Zeilennormalform} bringen
\item fehlende Zeilen (bei Sprung in ZNF) einfügen mit (-1) an entsprechender Stelle (sonst Nullen)
\item Kern ist linearer Aufspann der \textbf{Spalten} in denen die ergänzte(n) (-1) Stehen
\end{enumerate}
\subsection{Bild einer Matrix}
\begin{center}
\framebox[180pt]{$Bild A := \{ A\vec{x} : \vec{x} \in \mathbb{K}^m \} \subseteq \mathbb{K}^n$}
\end{center}
\begin{itemize}
\item entspricht der Lösungsmenge der Gleichung $A\vec{x} = \vec{b}$
\item $A\vec{x} = \vec{b}$ hat Lösung $\Leftrightarrow b \in$ Bild(A)
\item Bild(A) ist Unterraum von $K^n$ (genauer: linearer Aufspann der Spalten von A)
\end{itemize}
\subsection{Dimensionsformel}
\begin{center}
\framebox[250pt]{$dim(Kern(A)) + dim(Bild(A)) = m mit A\in \mathbb{K}^{n \ x \ m}$}
\end{center}
\subsection{Matrizenprodukt}
\begin{itemize}
\item $(\alpha A + \beta B)C = \alpha AC + \beta BC$
\item $(AB)C = A(BC)$
\item $AB \neq BA $
\end{itemize}
\newpage
\subsection{Inverse Matrix}
\begin{center}
\framebox[160pt]{$B=A^{-1}$ wenn $AB = BA = I$}
\end{center}
Die Matrix ist invertierbar (regulaer)\\
$\Leftrightarrow$ AB = I\\
$\Leftrightarrow$ Kern(A) = {0}\\
$\Leftrightarrow$ Rang(A) = dim(Bild(A)) = n
\end{document}
