

\documentclass[a4paper,12pt]{article} % The default font size is 12pt on A4 paper, change to 'usletter' for US Letter paper and adjust margins in structure.tex
\usepackage{amssymb}
\usepackage{fancyhdr}
\fancyhf{}
\fancyhead[L]{HM1 Tutorium 5}
\fancyhead[R]{22.11.2018}
\fancyfoot[L]{Moritz Luca Schmid}
\fancyfoot[R]{\thepage}
\renewcommand{\headrulewidth}{1pt}
\renewcommand{\footrulewidth}{1pt}
\fancypagestyle{plain}{}
\pagestyle{fancy}
\setlength{\headheight}{25pt}
\addtolength{\headwidth} {23mm}

%%%%%%%%%%%%%%%%%%%%%%%%%%%%%%%%%%%%%%%%%
% Contract
% Structural Definitions File
% Version 1.0 (December 8 2014)
%
% Created by:
% Vel (vel@latextemplates.com)
% 
% This file has been downloaded from:
% http://www.LaTeXTemplates.com
%
% License:
% CC BY-NC-SA 3.0 (http://creativecommons.org/licenses/by-nc-sa/3.0/)
%
%%%%%%%%%%%%%%%%%%%%%%%%%%%%%%%%%%%%%%%%%

%----------------------------------------------------------------------------------------
%	PARAGRAPH SPACING SPECIFICATIONS
%----------------------------------------------------------------------------------------

\setlength{\parindent}{0mm} % Don't indent paragraphs

\setlength{\parskip}{2.5mm} % Whitespace between paragraphs

%----------------------------------------------------------------------------------------
%	PAGE LAYOUT SPECIFICATIONS
%----------------------------------------------------------------------------------------

\usepackage{geometry} % Required to modify the page layout

\setlength{\textwidth}{16cm} % Width of the text on the page
\setlength{\textheight}{24.5cm} % Height of the text on the page

\setlength{\oddsidemargin}{0cm} % Width of the margin - negative to move text left, positive to move it right

% Uncomment for offset margins if the 'twoside' document class option is used
%\setlength{\evensidemargin}{-0.75cm} 
%\setlength{\oddsidemargin}{0.75cm}

\setlength{\topmargin}{-1.25cm} % Reduce the top margin

%----------------------------------------------------------------------------------------
%	FONT SPECIFICATIONS
%----------------------------------------------------------------------------------------

% If you are running Apple OS X, uncomment the next 4 lines and comment/delete the block below, you will now need to compile with XeLaTeX but your document will look much better

%\usepackage[cm-default]{fontspec}
%\usepackage{xunicode}

%\setsansfont[Mapping=tex-text,Scale=1.1]{Gill Sans}
%\setmainfont[Mapping=tex-text,Scale=1.0]{Hoefler Text}

%-------------------------------------------

\usepackage[utf8]{inputenc} % Required for including letters with accents
\usepackage[T1]{fontenc} % Use 8-bit encoding that has 256 glyphs

\usepackage{avant} % Use the Avantgarde font for headings
\usepackage{mathptmx} % Use the Adobe Times Roman as the default text font together with math symbols from the Sym­bol, Chancery and Com­puter Modern fonts

%----------------------------------------------------------------------------------------
%	SECTION TITLE SPECIFICATIONS
%----------------------------------------------------------------------------------------

\usepackage{titlesec} % Required for modifying section titles

\titleformat{\section} % Customize the \section{} section title
{\sffamily\large\bfseries} % Title font customizations
{\thesection} % Section number
{16pt} % Whitespace between the number and title
{\large} % Title font size
\titlespacing*{\section}{0mm}{7mm}{0mm} % Left, top and bottom spacing around the title

\titleformat{\subsection} % Customize the \subsection{} section title
{\sffamily\normalsize\bfseries} % Title font customizations
{\thesubsection} % Subsection number
{16pt} % Whitespace between the number and title
{\normalsize} % Title font size
\titlespacing*{\subsection}{0mm}{5mm}{0mm} % Left, top and bottom spacing around the title % Input the structure.tex file which specifies the document layout and style


\begin{document}
\section{Konvergenz von Folgen}
\subsection{Definition der Konvergenz}
\begin{center}
\framebox[200pt]{$\forall \epsilon > 0, \exists n_{0} \in \mathbb{N}, \forall n \geq n_{0} : |a_{n}-a| < \epsilon$}
\end{center}
\subsection{Folgerungen}
\begin{itemize}
\item $a_n$ ist konvergent, wenn $\lim\limits_{n \rightarrow \infty}{a_n} = a$ existiert (a ist Grenzwert)
\item Die Folge ist eine Nullfolge, wenn $a = 0$
\item Konvergente Folgen sind immer beschränkt $(|a_n| \leq c) \rightarrow limsup/liminf$
\item Konvergente Folgen haben nur einen Häufungswert (jede Teilfolge konvergiert gegen a)
\item konvergent, wenn monoton (wachsend/fallend) und beschränkt
\item \textbf{Satz von Bolanzo-Weierstraß:} Jede beschränkte Folge hat eine konvergente Teilfolge und damit mindestens einen Häufungswert
\item Folge hat zwei unterschiedliche Häufungswerte $\Rightarrow$ Divergenz
\item $a_n (konvergent) + b_n (divergent) \Rightarrow a_n + b_n (divergent)$
\end{itemize}
\subsection{Rechnen}
Egal was für ein Verfahren man für den Nachweis der Konvergenz nutzt, ist es wichtig, dass man alle n in der Folge \textbf{gleichzeitig} gegen unendlich gehen lässt.
\begin{itemize}
\item \textbf{Grenzwertsätze} zur Vereinfachung nutzen \\
($a_n+b_n \rightarrow a+b; \ \ a_n*b_n \rightarrow a*b; \ \ \frac{a_n}{b_n} \rightarrow \frac{a}{b}$ für $b_n \neq 0 \ \ und \ \ b \neq 0$)
\item $a_n \rightarrow \inf$ und $\frac{1}{a_n} \rightarrow 0$
\item $a_n + b_n \rightarrow a+b$ und $a_n * b_n \rightarrow a*b$
\item $\lim\limits_{n \rightarrow \infty}{\frac{1}{n}=0}$
\item $\sqrt[n]{n}\rightarrow 1$ aber $\sqrt[n]{n!}\rightarrow \inf$
\item $(1+\frac{1}{n})^n\rightarrow e$
\item $z^n$ konvergiert für $|z| < 1$ gegen 0 
\end{itemize}
\newpage
\subsection{Trickkiste}
\begin{itemize}
\item \textbf{Summe 0 Ergänzung:} z.B. +1-1 (um bestimmte Form zu erzwingen)
\item \textbf{Teilfolgenkriterium:} Wenn jede Teilfolge gegen a konvergiert, konvergiert auch die gesamte Folge gegen a. (a ist einziger Häufungspunkt)
\item "Monoton und beschränkt gibt konvergent."
\item \textbf{Sandwichkriterium:} $a_n \rightarrow a, c_n \rightarrow a$ und $a_n \leq b_n \leq c_n$ gilt auch $b_n \rightarrow a$
\item \textbf{Kürzungstrick:} bei Quotient mit höchstem Exponenten von n teilen
\item \textbf{3. Binomische Formel:} $(a-b)*(a+b) = a^2-b^2$ (bei Termen mit Wurzel)
\item $a_n$ mit $e^{ln(a_n)}$ erweitern $\rightarrow$ Logarithmusgesetze anwenden
\item wenn n als Exponent in Nenner und Zähler (z.B. $\frac{n^n}{(n+1)^n} \Rightarrow (\frac{n}{n+1})^n = (\frac{1}{1+\frac{1}{n}})^n$)
\item Bruch umdrehen: $\frac{x}{y} = \frac{1}{\frac{y}{x}}$
\end{itemize}
\section{Wichtige Abschätzungen und Gleichungen}
\begin{enumerate}
\item \textbf{Bernoullische Ungleichung:} $(1+x)^n \geq 1+n*x$
\item \textbf{geometrische Summenformel:} $u^m-v^m = (u-v)*\sum_{k=0}^{m-1}u^{m-1-k}  v^k$
\item \textbf{Binominalsatz:} $(a+b)^n=\sum_{k=0}^{n} {n \choose k} a^kb^{n-k}$
\item \textbf{Binominal Koeffizienten:} ${n \choose k} = \frac{n!}{k!(n-k)!}$
\end{enumerate}
\end{document}
