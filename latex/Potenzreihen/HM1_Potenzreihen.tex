

\documentclass[a4paper,12pt]{article} % The default font size is 12pt on A4 paper, change to 'usletter' for US Letter paper and adjust margins in structure.tex

\usepackage{fancyhdr}
\fancyhf{}
\fancyhead[L]{HM1 Tutorium 7}
\fancyhead[R]{13.12.2018}
\fancyfoot[L]{Moritz Luca Schmid}
\fancyfoot[R]{\thepage}
\renewcommand{\headrulewidth}{1pt}
\renewcommand{\footrulewidth}{1pt}
\fancypagestyle{plain}{}
\pagestyle{fancy}
\setlength{\headheight}{25pt}
\addtolength{\headwidth} {23mm}

%%%%%%%%%%%%%%%%%%%%%%%%%%%%%%%%%%%%%%%%%
% Contract
% Structural Definitions File
% Version 1.0 (December 8 2014)
%
% Created by:
% Vel (vel@latextemplates.com)
% 
% This file has been downloaded from:
% http://www.LaTeXTemplates.com
%
% License:
% CC BY-NC-SA 3.0 (http://creativecommons.org/licenses/by-nc-sa/3.0/)
%
%%%%%%%%%%%%%%%%%%%%%%%%%%%%%%%%%%%%%%%%%

%----------------------------------------------------------------------------------------
%	PARAGRAPH SPACING SPECIFICATIONS
%----------------------------------------------------------------------------------------

\setlength{\parindent}{0mm} % Don't indent paragraphs

\setlength{\parskip}{2.5mm} % Whitespace between paragraphs

%----------------------------------------------------------------------------------------
%	PAGE LAYOUT SPECIFICATIONS
%----------------------------------------------------------------------------------------

\usepackage{geometry} % Required to modify the page layout

\setlength{\textwidth}{16cm} % Width of the text on the page
\setlength{\textheight}{24.5cm} % Height of the text on the page

\setlength{\oddsidemargin}{0cm} % Width of the margin - negative to move text left, positive to move it right

% Uncomment for offset margins if the 'twoside' document class option is used
%\setlength{\evensidemargin}{-0.75cm} 
%\setlength{\oddsidemargin}{0.75cm}

\setlength{\topmargin}{-1.25cm} % Reduce the top margin

%----------------------------------------------------------------------------------------
%	FONT SPECIFICATIONS
%----------------------------------------------------------------------------------------

% If you are running Apple OS X, uncomment the next 4 lines and comment/delete the block below, you will now need to compile with XeLaTeX but your document will look much better

%\usepackage[cm-default]{fontspec}
%\usepackage{xunicode}

%\setsansfont[Mapping=tex-text,Scale=1.1]{Gill Sans}
%\setmainfont[Mapping=tex-text,Scale=1.0]{Hoefler Text}

%-------------------------------------------

\usepackage[utf8]{inputenc} % Required for including letters with accents
\usepackage[T1]{fontenc} % Use 8-bit encoding that has 256 glyphs

\usepackage{avant} % Use the Avantgarde font for headings
\usepackage{mathptmx} % Use the Adobe Times Roman as the default text font together with math symbols from the Sym­bol, Chancery and Com­puter Modern fonts

%----------------------------------------------------------------------------------------
%	SECTION TITLE SPECIFICATIONS
%----------------------------------------------------------------------------------------

\usepackage{titlesec} % Required for modifying section titles

\titleformat{\section} % Customize the \section{} section title
{\sffamily\large\bfseries} % Title font customizations
{\thesection} % Section number
{16pt} % Whitespace between the number and title
{\large} % Title font size
\titlespacing*{\section}{0mm}{7mm}{0mm} % Left, top and bottom spacing around the title

\titleformat{\subsection} % Customize the \subsection{} section title
{\sffamily\normalsize\bfseries} % Title font customizations
{\thesubsection} % Subsection number
{16pt} % Whitespace between the number and title
{\normalsize} % Title font size
\titlespacing*{\subsection}{0mm}{5mm}{0mm} % Left, top and bottom spacing around the title % Input the structure.tex file which specifies the document layout and style


\begin{document}

\section{Potenzreihen}
Mit Potenzreihen kann man reellwertige Funktionen beschreiben. Sie spielen eine sehr wichtige Rolle in der Numerik und auch in der Elektrotechnik (z.B. in Signale und Systeme).
\begin{center}
\framebox[100pt]{$\Sigma_{n=0}^{\infty} a_n (z-z_0)^n$}
\end{center}
\begin{itemize}
\item \textbf{$a_n$:} Koeffizient
\item \textbf{$z_0$:} Entwicklungspunkt
\item \textbf{Konvergenzradius R:} für $|z-z_0| < R$ konvergiert die Reihe (Kreis mit Radius R um Entwicklungspunkt $z_0$); der Fall $|z-z_0| = R$ muss extra untersucht werden
\item \textbf{Berechung des Konvergenzradius: }\\
\begin{center}
\framebox[200pt]{$R:= \frac{1}{limsup \sqrt[n]{|a_n|}} \ \ =  \ \ lim |\frac{a_n}{a_{n+1}}|$}
\end{center}
\end{itemize}

\subsection{Wichtige Potenzreihen}
\begin{itemize}
\item \textbf{Exponentialfunktion: } $e^x = \Sigma_{n=0}^{\infty}\frac{x^n}{n!}$
\item \textbf{Sinus: } $Sin(x) = \Sigma_{n=0}^{\infty}(-1)^n\frac{x^{2n+1}}{(2n+1)!}$
\item \textbf{Cosinus: } $Cos(x) = \Sigma_{n=0}^{\infty}(-1)^n\frac{x^{2n}}{(2n)!}$
\item \textbf{Logarithmus: } $ln(1+x) = \Sigma_{n=1}^{\infty}(-1)^{n+1}\frac{x^n}{n}$
\item \textbf{geometrische Reihe: } $\frac{1}{1-x} = \Sigma_{n=0}^{\infty} x^n$ \ \ \ (für $|x| \leq 1$)
\item \textbf{Ableitung geom. Reihe: } $\frac{1}{(1-x)^2} = \Sigma_{n=0}^{\infty} (n+1) x^n$  \ \ \ (für $|x| \leq 1$)
\end{itemize}

\subsection{Rechnen}
\begin{itemize}
\item \textbf{Cauchy-Produkt:} $(\Sigma \ \ a_n (z)^n) (\Sigma \ \ b_n (z)^n) \ \ = \ \ \Sigma_{k=0}^{n} \ \ a_k \ \ b_{n-k} $
\end{itemize}
\subsection{Tricks}
\begin{itemize}
\item Ableitung einer Potenzreihe hat den gleichen Konvergenzradius R
\item Teile, die unabhängig von n sind, können aus der Summe herausgezogen werden
\item Substitution (z.B. bei $x^n x^n$ substituiere $|x| = \sqrt{y} $)
\item Reihe in bekannte Reihe (oft die geom. Reihe, R = 1) umformen
\item Partialbruchzerlegung
\end{itemize}
\end{document}
