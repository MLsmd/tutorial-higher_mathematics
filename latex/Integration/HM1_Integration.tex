

\documentclass[a4paper,12pt]{article} % The default font size is 12pt on A4 paper, change to 'usletter' for US Letter paper and adjust margins in structure.tex
\usepackage{amssymb} %mathbb{}
\usepackage{fancyhdr} %\framebox[150pt]{}
\fancyhf{}
\fancyhead[L]{HM1 Tutorium 13}
\fancyhead[R]{31.01.2019}
\fancyfoot[L]{Moritz Luca Schmid}
\fancyfoot[R]{\thepage}
\renewcommand{\headrulewidth}{1pt}
\renewcommand{\footrulewidth}{1pt}
\fancypagestyle{plain}{}
\pagestyle{fancy}
\setlength{\headheight}{25pt}
\addtolength{\headwidth} {23mm}

\input{../structure.tex} % Input the structure.tex file which specifies the document layout and style


\begin{document}
\section{Integration}
Das Integral ist die "Summe einer kontiniuierlichen Funktion". Im eindimensionalen Fall entspricht das Integral über die Funktion f(x) der Fläche unter dem Graphen von f.

\subsection{Partielle Integration}
\begin{center}
\framebox[150pt]{$\int f' g \ dx \ = \ f g \ - \ \int f g'$}
\end{center}
Die partielle Integration ist dann sinnvoll, \textbf{wenn ein störendes Element durch ableiten verschwindet} (z.B. x' = 1).
\begin{itemize}
\item \textbf{"Faktor 1 Trick":} Wenn eigentliche Funktion nicht aufleitbar\\
$\rightarrow$ Funktion mit 1 multiplizieren (f'=1 und g= Funktion), somit \textbf{muss die Funktion nur abgeleitet}, aber nicht aufgeleitet werden.\\
(z.B. $ln(x) = 1*ln(x) \Rightarrow \int 1*ln(x) dx \ = \ x*ln(x) \ - \ \int x*\frac{1}{x} \ = ...$)
\item \textbf{"Faktor 2 Trick":} Wenn im neuen Integral der part. Integration das selbe wie im eigentlichen Integral steht (vor allem bei sin/cos/exp):\\
z.B. $\int sin \ cos \ dx \ = \ -cos^2 + c \ - \ \int sin \ cos dx \  \ \ \ \ |auf beiden Seiten +\int sin \ cos \ dx$ und $*\frac{1}{2}$\\
$= \frac{1}{2} cos^2 + \frac{c}{2}$
\end{itemize}

\subsection{Substitution}
\begin{center}
\framebox[290pt]{(i) $\int f(u(x)) \ u'(x) \ dx \ = \ F(u(x)) + c \ \ \ \ \ \ \  \ \ \ \ \ \ \ \ \ \ \ \ \ \ \ \ \ \ \ \ \ \ \ \ \ \ $ } \\
\framebox[290pt]{(ii)$I=[a,b] \Rightarrow \int_{a}^b f(u(x)) \ u'(x) \ dx \ = \ F(u(b)) - F(u(a))$ }
\end{center}
Bei unschöner Verkettung einfach die Verkettung substituieren. Oftmals geben auch die Integralgrenzen Aufschluss darüber, welche Substitution geschickt ist.\\
\textbf{Wichtig:} Bei unbestimmtem Integral rücksubstituieren; sonst Grenzen anpassen. \textbf{Nicht vergessen!!}

\subsection{Uneigentliches Integral}
Eine Grenze des Integrals ist $\infty$ bzw. lässt den Nenner eines Quotienten 0 setzen. (z.B. $\int_{0}^{\infty} x^2 \ dx$
Die "kritische" Integralgrenze wird mit einer Variablen a ersetzt. Nach der Berechnung des Integrals berechnet man den Grenzwert (limes) des Terms.
\begin{itemize}
\item \textbf{konvergent}, wenn $lim_{a \rightarrow \xi}$ existiert
\end{itemize}
\subsection{Tricks zur Analyse von uneigentlichen Integralen}
\begin{itemize}
\item \textbf{Partialbruchzerlegung }($\frac{1}{x*(x+4)} \ = \ \frac{A}{x} + \frac{B}{x+4}$)
\item \textbf{Majorantenkriterium für Konvergenz (bzw. Minorante für Divergenz)}
\item Integral aufteilen, wenn beide Grenzen kritisch ($\int_{-\infty}^\infty x \ dx  \ = \ \int_{-\infty}^c \ dx + \int_{c}^\infty x \ dx$)
\end{itemize}

\end{document}
